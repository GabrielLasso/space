%        File: Manual.tex
%     Created: Qui Jun 09 06:00  2016 B
% Last Change: Qui Jun 09 06:00  2016 B
%
\documentclass[a4paper]{book}
\usepackage[brazilian]{babel}
\title {Space}
\author{
    Mathias Van Sluys Menck 
    \texttt{4343470}
    \and
    Gabriel Kuribara Lasso 
    \texttt{9298016}
    \and
    Rodrigo Zanette de Magalhães 
    \texttt{9298090}
}
\date { }

\begin{document}
\maketitle
\tableofcontents

\chapter {Comecando}

\section {Instalacao}
Para instalar, va ate o diretorio em que o jogo se encontra e rode o comando

make

\section {Abrindo o jogo}

No diretorio em que o jogo foi instalado, abra o executavel space pelo terminal usando o comando 

./space

Entao escolha a nave de cada jogador e a fase digitando o numero correspondente

\section {Controles}
Para controlar a nave do jogador 1, use as teclas das setas, e para o jogador 2, use WASD.
\\\\Comandos:
\\Cima/W: Acelera a nave
\\Esquerda/A: Gira a nave no sentido anti-horario
\\Direita/D: Gira a nave no sentido horario
\\Baixo/S: Atira

\chapter {Elementos do jogo}

\section {Naves}
Todas as naves possuem atributos de massa, pontos de vida, velocidade de atiramento, de rotacao, de movimento e raio de colisao

\subsection {Hyperion}
Uma fortaleza espacial de grande porte, resiste aos ataques como apenas os muros mais densos o fazem.
\\Massa: 3000
\\Vida: 5
\\Velocidade de tiros: 3 tiros por segundo
\\Velocidade de rotacao: 0.42 voltas por segundo
\\Velocidade de movimento: 0.01
\\Raio de colisao: 50 pixels

\subsection {Aegis}
Uma nave basica e para aqueles que não buscam excelencia, mas equilibrio.
\\Massa: 1500
\\Vida: 2
\\Velocidade de tiros: 4 tiros por segundo
\\Velocidade de rotacao: 0.54 voltas por segundo
\\Velocidade de movimento: 0.12
\\Raio de colisao: 29 pixels

\subsection {Tartarus}
Uma nave agil e potente, feita com precisão de controle em mente.
\\Massa: 850
\\Vida: 2
\\Velocidade de tiros: 4.62 tiros por segundo
\\Velocidade de rotacao: 1.87 voltas por segundo
\\Velocidade de movimento: 0.1
\\Raio de colisao: 25 pixels

\subsection {Argo}
Inspirada em uma famosa fortaleza, esta nave consegue tanto receber quanto devolver golpes.
\\Massa: 2000
\\Vida: 3
\\Velocidade de tiros: 4 tiros por segundo
\\Velocidade de rotacao: 0.37 voltas por segundo
\\Velocidade de movimento: 0.07
\\Raio de colisao: 37 pixels

\subsection {Melpomene}
Nave perigosa tanto para o inimigo quanto o piloto, ela é desenhada para ataques relampagos e missoes suicidas.
\\Massa: 700
\\Vida: 1
\\Velocidade de tiros: 7.5 tiros por segundo
\\Velocidade de rotacao: 0.94 voltas por segundo
\\Velocidade de movimento: 0.13
\\Raio de colisao: 17 pixels

\section {Fases}

\subsection {Bidres}
Uma fase rosa com um planeta feito de gelatina.

\begin {quote}
    Yummy!!
\end {quote}

\subsection {Quater}
Tipo um buraco negro, so que ao contrario. Ele vai te jogar para longe.

\subsection {Octans}
Um planeta grande e azul;

\subsection {Hexadecyon}
Um planeta que, embora pequeno, e' muito pesado.

\chapter {Customizando}
Tambem e possivel adicionar novas naves e fases, alem de modificar as que ja existem. Para isso, basta seguir os passos a seguir:

\section {Adicionando naves}
No diretorio resources/naves/ ha uma pasta para cada nave. Para adicionar uma nave nova, basta criar uma pasta com os arquivos corretos.
\\Os arquivos necessarios para uma nave sao:

- 16 imagens no formato .xpm nomeadas de 00.xpm a 15.xmp, cada uma para um angulo da nave, sendo a primeira com ela virada para cima e as demais indo no sentido horario.

- Um arquivo chamado proprieties contendo seis numeros:

A massa da nave

O tempo entre os tiros da nave (em frames);

O tempo entre as rotacoes (em frames, cada rotacao e de 22,5º)

A vida da nave;

O diametro usado para calcular as colisoes (em pixels);

A velocidade da nave (recomenda-se um valor pequeno)

\section {Adicionando fases}
De forma semelhante, no diretorio resources/fases/ ha uma pasta para cada fase. Para adicionar uma fase nova, basta criar uma pasta com os arquivos corretos.
\\Os arquivos necessarios para uma fase sao:

- Uma imagem bg.xpm que sera a imagem de fundo

- Uma imagem planeta.xpm que sera a imagem do planeta

- Um arquivo chamado newgame contendo as informacoes da fase organizados da seguinte forma:
\\D M
\\X1 Y1 Vx1 Vy1
\\X2 Y2 Vx2 Vy2
\\W H
\\
\\Em que:
\\D e o diametro do planeta
\\M e a massa do planeta
\\(X1, Y1) sao as coordenadas iniciais do jogador 1
\\(Vx1, Vy1) e a velocidade inicial do jogador 1
\\(X2, Y2) sao as coordenadas iniciais do jogador 2
\\(Vx2, Vy2) e a velocidade inicial do jogador 2
\\W e a largura da janela em pixels
\\H e a altura da janela em pixels

\end{document}

